% Dedekind-finite rings

Ассоциативное кольцо $ R $ с двусторонней единицей называется \textit{Дедекинд-конечным}, если для любых двух его элементов $ a $ и $ b $ равенство $ ab = 1 $ влечёт $ ba = 1 $.

\begin{theorem}
	Следующие условия на ассоциативное кольцо $ R $ с двусторонней единицей эквивалентны:
	\begin{enumerate}
		\item $ R $ --- Дедекинд-конечно;
		\item в $ R $ существует элемент $ a $ являющийся одновременно левым делителем 1 и левым делителем 0;
		\item в $ R $ существует элемент $ b $ являющийся одновременно правым делителем~1 и правым делителем 0.
	\end{enumerate}
\end{theorem}

\begin{proof}
	\hfill
	
	\textbf{[$ 1 \Rightarrow 2, 3 $]}
	
	Пусть $ ab = 1 $ и $ ba \neq 1 $.
	Тогда $ ba - 1 \neq 0 $ и $ a(ba - 1) = aba - a = (ab - 1)a = 0 $.
	Аналогично $ (ba - 1)b = bab - b = b(ab - 1) = 0 $.
	
	\textbf{[$ 2 \Rightarrow 1 $]}
	
	Пусть $ ab = 1 $ и $ ac = 0 $, $ c \neq 0 $.
	Если бы было выполнено $ ba = 1 $, то $ c =  bac = b0 = 0 $, что противоречило бы предположению.
	
	\textbf{[$ 3 \Rightarrow 1 $]}
	
	Следует из предыдущего для $ \op{R} $.
	
\end{proof}