% Simple groups: Alternative Groups

Группа называется \textcolor{magenta}{простой}, если в ней нет собственных нормальных подгрупп.

%Приведем несколько примеров простых групп:

\begin{theorem}\label{Tm_An_simple}
Группы \(A_{n}\) при \(n \neq 4\) просты.
\end{theorem}
\textbf{Доказательство.}
\footnote{Сделано на основе лекций по алгебре, читавшихся в Красноярском государственном университете}
Группы $A_1, A_2, A_3$ имеют порядки 1, 1, 3 соответственно, а потому просты. 
\medskip

%Группа \(A_{3} = \left\langle \left( {123} \right) \right\rangle_{3}\) имеет порядок 3. Следовательно, она не имеет нетривиальных подгрупп и поэтому проста.

Группа \(A_{4}\) содержит четверную группу Клейна \(V_{4}=\{\Id , (12)(34), (13)(24), (14)(23) \}\), которая нормальна в $A_4$. Следовательно, $A_4$ не проста.
\medskip

Таким образом, теорему необходимо доказать при $n\geqslant5$.
\medskip

Дальнейшее доказательство теоремы опирается на ряд следующих вспомогательных утверждений.

\begin{theorem}\label{Tm_An_Generation}
Группа \(A_n\) порождается тройными циклами.
\end{theorem}

\textbf{Доказательство.}
%
Мы знаем, что каждая подстановка есть произведение циклов длины 2
(транспозиций). Так как подстановки в \(A_{n}\) четны, то они равны
произведению четного числа транспозиций. Рассмотрим различные (по числу общих символов) варианты произведения двух
соседних транспозиций:

\( (ij)(kl) = (ijk)(jkl)\),
если \(i,j,k,l\) все различны (нуль общих символов),

\(\left( ij \right)\left( jk \right) = \left( {{ijk}} \right)\),
если \(i,j,k\) различны (один общий символ), и, наконец,

\(\left( {{ij}} \right)\left( {{ij}} \right) = 1\).
\medskip

Таким образом, сгруппировав транспозиции по две, мы сможем представить произведение четного числа транспозиций как произведение циклов
длины 3. $\square$

\begin{lemma}\label{Lm_If_triple}
Пусть $N$ нормальная подгруппа в \(A_{n}\), $n\geqslant 5$. Если \(N\) содержит тройной цикл \( (ijk) \in N\), то \(N = A_{n}\).
\end{lemma}

\textbf{Доказательство.}
%
Возьмём произвольный тройной цикл \(\left( {{abc}} \right)\),
возьмём подстановку
\(\sigma = \begin{pmatrix}
i & j & k & u & v & \ldots \\
a & b & c & u' & v' & \ldots \\
\end{pmatrix}\), такую что
\(u', v' \neq a, b, c\) и \(u', v'\in\{i,j,k,u,v\} \), далее все элементы
переходят в себя.
При необходимости меняем $u', v'$ местами: одна из таких подстановок будет чётной, выберем ее.
Получаем
\(\sigma\left( {{ijk}} \right)\sigma^{-1} = \left( {{abc}} \right) \in N\),
так как \(N \vartriangleleft A_{n}\). Следовательно, подгруппе \(N\)
принадлежат все тройные циклы. Отсюда (по предыдущей лемме)
\(N = A_{n}\). $\square$

\begin{lemma}
Если \(N\vartriangleleft A_n\) и содержит подстановку \(\sigma\), у которой в разложении на независимые циклы встречается цикл длины \(\geqslant4\), то \(N = A_{n}\).
\end{lemma}
\textbf{Доказательство.}
%
Пусть \(\sigma = \left( {{ijkl}}\ldots \right)\ldots\in N\).
Тогда
\(\tau = \underset{\in N}{\underbrace{(ijk)\sigma(ijk)^{-1}}}\underset{\in N}{\underbrace{\sigma^{-1}}} = (ijl) \in N\),
то есть \(N\) содержит цикл длины 3.
Следовательно (по лемме о порождении $A_n$), \(N = A_{n}\).
$\square$

\begin{lemma}
Если \(N\vartriangleleft A_n\) и содержит подстановку \(\sigma\), у которой в разложении на независимые циклы встречается цикл длины \(\geqslant3\), а также еще какие-либо нетривиальные циклы, то \(N = A_{n}\).
\end{lemma}
\textbf{Доказательство.}
Пусть \(\sigma = (ijk)(lm\ldots)\ldots\in N\).
Тогда
%\(\tau = \underset{\in N}{\underbrace{\sigma^{-1}}}\underset{\in N}{\underbrace{(jkl)^{-1}\sigma(jkl)}} = \left(iljkm\right) \in N\).
\(\tau = \underset{\in N}{\underbrace{(jkl)\sigma(jkl)^{-1}}} \underset{\in N}{\underbrace{\sigma^{-1}}} = \left(iljkm\right) \in N\).
Следовательно (по предыдущей лемме),
\(N = A_{n}\). $\square$

\begin{lemma}
Если \(N\vartriangleleft A_n\ (n\geqslant5)\) и содержит подстановку \(\sigma\), у которой в разложении на независимые циклы содержатся только циклы длины 2, то \(N = A_{n}\).
\end{lemma}

\textbf{Доказательство.}
%
Если
\(\sigma = \left( {{ij}} \right)\left( {{ab}} \right)\),
то, так как у нас не менее пяти символов,
\(\exists c \notin \left\{ i,j,a,b \right\}\). Тогда
\(\tau = \left( {{ijc}} \right)\sigma\left( {{ijc}} \right)^{-1}\sigma^{-1} = \left( {{icj}} \right) \in N\),
следовательно (по лемме \ref{Lm_If_triple}), \(N = A_{n}\).

Если
\(\sigma = \left( {{ij}} \right)\left( {{ab}} \right)\left( {{uv}} \right)\left( {{pq}} \right)\ldots\),
то
\((ja)(bu)\sigma(bu)\left( {{ja}} \right)\sigma^{-1} = \left( {{iub}} \right)\left( {{jav}} \right) \in N\),
следовательно (по предыдущей лемме), \(N = A_{n}\).
$\square$
\medskip

Теперь, собственно, докажем теорему \ref{Tm_An_simple}. Возьмём произвольную нетривиальную подстановку
\(\sigma \in N\). Она удовлетворяет условию одной из предыдущих лемм, следовательно \(N = A_{n}\). Теорема доказана. $\square$
\medskip
