% Теорема Силова

Пусть $ G $ --- конечная группа, $ p $ --- простое число и $ |G| = p^km $, где $ m $ взаимно просто с $ p $.
Подгруппа $ H \leqslant G $, порядок которой равен $ p^k $ называет \textit{силовской $ p $-подгруппой} группы $ G $.

\begin{theorem}[Силов]
	Пусть $ G $ --- конечная группа, $ p $ --- простое число и $ |G| = p^km $, где $ m $ не кратно $ p $.
	Тогда
	\begin{enumerate}
		\item для каждого $ i $ от $ 1 $ до $ k $ существует подгруппа порядка $ p^i $ в группе $ G $;
		
		\item для всех $ i $, кроме $ k $, всякая подгруппа порядка $ p^{i} $ 
		вложена в подгруппу порядка $ p^{i + 1} $;
		
		\item все подгруппы порядка $ p^k $ в группе $ G $ сопряжены;
		
		\item количество подгрупп порядка $ p^k $ сравнимо с 1 по модулю $ p $ и делит число $ m $.
	\end{enumerate}
\end{theorem}

\begin{proof}
	\hfill
	
	\textbf{[1, существование]} 
	
	Пусть $ |G| = p^km $.
	Пусть $\mathcal{X}$ --- множество всех подмножеств мощности $p^i$ из $G$.
	Тогда
	
	$$ |\mathcal{X}| = C_{p^km}^{p^i} = 
	\tfrac{(p^km)!}{p^i! (p^km - p^i)!} =
	\tfrac{p^km}{p^i} \prod\limits_{j = 1}^{p^i - 1} \tfrac{p^km - j}{j}. $$ 
	
	Если $ X \in \mathcal{X} $ и $ g \in G $, то $ gM = \defineset{gm}{m \in M} \in \mathcal{M} $.
	Так что $ G $ действует на $\mathcal{M}$ левыми сдвигами.
	В формуле, выражающей мощность $ \mathcal{X} $ 
	ни один множитель в произведении справа не делится на $ p $, поэтому
	наибольшая степень $ p $, делящая $ |\mathcal{X}| $, --- это $ p^{k-i} $. 
	Найдём орбиту $ \{X_1, \ldots, X_s\} $, мощность $ s $ которой не делится на $p^{k-i+1}$. 
	Положим 
	$$ G_i= \defineset{g \in G}{gX_1 = X_i}, 1\leqslant i\leqslant s. $$
	Множество $ G_1 $ --- стабилизатор $ X_1 $ и, следовательно, 
	подгруппа в $ G $, 
	а $ G_i $ --- левые смежные классы $ G $ по $ G_1 $. 
	
	Покажем, что подгруппа $ G_1 $ имеет требуемый порядок $p^i$.
	Действительно, $ |G_1| \cdot s = |G| = p^km $.
	Так как $ s $ не делится на $ p^{k - i + 1} $,
	то $ |G_1| $ делится на $ p^i $ и поэтому $ |G_1| \geqslant p^i $.
	С другой стороны, возьмём $ x \in X_1 $. Тогда $ G_1x \subset X_1 $.
	Поскольку $ |X_1| = p^{i} $, то $ |G_1| = p^i $.
	
	\textbf{[2, вложенность]}
	
	Пусть $ p^{i+1} $ делит $ |G| $, $ P $ --- подгруппа порядка $ p^i $ из G, $ \mathcal{P} $~--- класс подгрупп, сопряжённых с $ P $ элементами из $ G $.
	Мы знаем, что $ |\mathcal{P}|=|G:N_G(P)| $. 
	
	Если $ |\mathcal{P}| $ не делится на $ p $, то $ |N_G(P)| $ делится на $ p^{i+1} $, 
	а потому по первой части пункта 1 теоремы в $ N_G(P)/P $ существует подгруппа $ P^\ast/P $ порядка $p$. Тогда $ P^\ast $ --- требуемая подгруппа $ G $.
	
	Пусть теперь $|\mathcal{P}|$ делится на $p$. Группа $ P $ действует на $\mathcal{P}$ сопряжениями, причём мощности орбит делят $|P|$,
	а потому имеют вид $p^{k_j}, k_j\geqslant 0$.
	Имеется по крайней мере одна одноэлементная орбита $\{P\}$ и $|\mathcal{P}|$ делится на $p$.
	Поэтому непременно найдётся и другая одноэлементная орбита $\{Q\}$. 
	Но это означает, что $ P $ нормализует $ Q $, поэтому $ PQ $ есть подгруппа и, более того, $p$-подгруппа.
	Последнее следует из того, что $|PQ| = |Q|\cdot |PQ/Q|$ и того, что 
	$ PQ/Q\simeq P/P\cap Q $.
	Применяя к $ PQ $ то сопряжение группы $ G $, которое переводит $ Q $ в $ P $, мы получим $p$-подгруппу $P'P$, содержащую $P$ в качестве собственной нормальной подгруппы.
	Снова по первой части теоремы в $P'P/P$ найдётся подгруппа $P^\ast/P$ порядка $p$, тогда $P^\ast$ --- требуемая подгруппа.

	\textbf{[3, 4]}
	
	Пусть $ S $ --- непустое множество подгрупп порядка $ p^k $, 
	инвариантное относительно действия группы $ G $ сопряжениями.
	Покажем, что порядок $ S $ сравним с 1 по модулю $ p $.
	
	Действительно, пусть $ H \in S $ --- подгруппа порядка $ p^k $.
	Рассмотрим действие $ H $ на $ S $ сопряжениями. Порядок всякой орбиты делит $ p^k $.
	Тогда либо порядок орбиты делится на $ p $, либо она состоит ровно из одной подгруппы.
	Если $ \{H'\} $ --- одноэлементная орбита, то $ H $ нормализует $ H' $.
	Тогда группа $ HH' $ содержится в $ N_G(H') $ и $ H' $ нормальна в $ HH' $.
	Отсюда $ |HH'| = |H'| \cdot |HH' / H'| = |H'| \cdot |H / H \cap H'| $ является степенью числа $ p $.
	Так как $ p^k $ --- набольшая степень $ p $, делящая порядок $ G $, то $ H = HH' = H' $.
	Таким образом, одноэлементная орбита ровно одна и $ |S| \equiv 1 \pmod p $.
	
	Из доказанного следует, что множеств подгрупп порядка $ p^k $, 
	инвариантных относительно сопряжения, не может быть двух.
	Иначе порядок объединения двух инвариантных подмножеств не был бы сравним с 1 по модулю $ p $.
	Тогда все подгруппы порядка $ p^k $ сопряжены, их количество сравнимо с 1 по модулю $ p $
	и делит порядок группы $ G $.
	
\end{proof}