% Теорема Жордана-Гёльдера

\begin{theorem}[Жордан, Гельдер]
	Пусть $ G $ --- конечная группа. Тогда существует композиционный ряд
	$$ G = G_0 \rhd G_1 \rhd \ldots \rhd G_n = \{e\}, $$
	причём число $ n $ и набор простых групп $ \{G_i/G_{i + 1}\}_{i = 0}^{n - 1} $ определены однозначно.
\end{theorem}

\begin{proof}
	\hfil
	
	\textbf{[существование]}
	
	Индукция по мощности группы $ G $.
	
	База: $ |G| = 1 $. Тогда $ G = G_0 = \{e\} $.
	
	Шаг. Если $ G $ проста, то возьмём ряд $ G = G_0 \rhd G_1 = \{e\} $.
	Иначе найдём в группе $ G $ максимальную нормальную подгруппу $ G_1 $, не совпадающую с $ G $.
	Группа $ G_0/G_1 $ проста, иначе её собственную нормальную подгруппу 
	можно было бы поднять до нормальной подгруппы в группе $ G $
	и $ G_1 $ не была бы максимальной.
	Поскольку порядок группы $ G_1 $ меньше порядка $ G_0 $,
	поэтому для неё композиционный ряд для $ G_1 $ существует по индукционному предположению. 
	
	\textbf{[единственность]}
	
	Индукция по порядку группы $ G $.
	
	База индукции: $ |G| = 1 $  или $ G $ проста. Тогда $ G = G_0 = \{e\} $ в первом случае и $ G = G_0 \rhd G_1 = \{e\} $ во втором случае.
	
	Шаг. Пусть имеются два композиционных ряда
	$$ G = G_0 \rhd G_1 \rhd \ldots \rhd G_n = \{e\}, $$
	$$ G = G_0 \rhd H_1 \rhd \ldots \rhd H_m = \{e\}. $$ 
	
	Если $ G_1 = H_1 $, то теорема следует из индукционного предположения для $ G_1 $ и $ H_1 $.
	Иначе положим $ K_2 = G_1 \cap H_1 $.
	Для группы $ K $ имеется композиционный ряд
	$$ K_2 \rhd K_3 \rhd \ldots \rhd K_r = \{e\}. $$
	Так как $ G_1 $ и $ H_1 $ нормальны в $ G $, то $ G_1H_1 $ --- нормальная подгруппа в $ G $.
	Так как $ G_1 $ и $ H_1 $ были максимальными нормальными подгруппами в $ G $,
	не совпадающими с $ G $ и не были тривиальными, то $ G_1H_1 = G $.
	
	По теореме об изоморфизме имеем
	$$ G / H_1 \cong G_1H_1 / H_1 \cong G_1 / (G_1 \cap H_1) \cong G_1 / K_2, $$  
	$$ G / G_1 \cong G_1H_1 / G_1 \cong H_1 / (G_1 \cap H_1) \cong H_1 / K_2. $$
	Тогда группы $ G_1 / K_2 $ и $ H_1 / K_2 $ просты.
	Поэтому имеются композиционные ряды
	$$ G_1 \rhd K_2 \rhd K_3 \rhd \ldots \rhd K_r = \{e\}, $$
	$$ H_1 \rhd K_2 \rhd K_3 \rhd \ldots \rhd K_r = \{e\}. $$
	По индукционному предположению наборы простых групп
	$$ \{G_1 / K_2, K_2 / K_3, \ldots,  K_{r - 1} / K_{r}\} =
	 \{G_1 / G_2, G_2 / G_3, \ldots, G_{n - 1} / G_{n}\} $$
	совпадают. Следовательно, $ r = n $.
	Аналогично, наборы простых групп
	$$ \{H_1 / K_2, K_2 / K_3, \ldots,  K_{r - 1} / K_{r}\}
	= \{H_1 / H_2, H_2 / H_3, \ldots, H_{m - 1} / H_{m}\} $$
	совпадают и $ n = r = m $.
	Тогда совпадают наборы
	$$ \{G / G_1, G_1 / G_2, G_2 / G_3, \ldots, G_{n - 1} / G_{n}\}
	= \{G / G_1, G_1 / K_2, K_2 / K_3, \ldots,  K_{r - 1} / K_{r}\} = $$
	$$ = \{G / H_1, H_1 / K_2, K_2 / K_3, \ldots,  K_{r - 1} / K_{r}\} 
	= \{G / H_1, H_1 / H_2, H_2 / H_3, \ldots, H_{m - 1} / H_{m}\}. $$
	Первое и третье равенство получаются из доказанных ранее добавлением
	$ G / G_1 $ и $ G / H_1 $ соответственно.
	Второе равенство верно, так как $ G / G_1 \cong H_1 / K_2 $
	и $ G / H_1 \cong G_1 / K_2 $.
	
	$$ \begin{tikzcd}
	&  G_1 \ar[symbol=\rhd]{r}{} \ar[symbol=\rhd]{dr}{} & G_2 \ar[symbol=\rhd]{r}{} 
	& G_3 \ar[symbol=\rhd]{r}{} & \ldots \ar[symbol=\rhd]{r}{} & G_n = \{e\} \\
	G_0 = G_1H_1 \ar[symbol=\rhd]{ur}{} \ar[symbol=\rhd]{dr}{} &  & G_1 \cap H_1 = K_2 \ar[symbol=\rhd]{r}{} 
	& K_3 \ar[symbol=\rhd]{r}{} & \ldots \ar[symbol=\rhd]{r}{} & K_r = \{e\} \\
	& H_1 \ar[symbol=\rhd]{ur}{} \ar[symbol=\rhd]{r}{} & H_2 \ar[symbol=\rhd]{r}{} 
	& H_3 \ar[symbol=\rhd]{r}{} & \ldots \ar[symbol=\rhd]{r}{} & H_m = \{e\}
	\end{tikzcd} $$
	
\end{proof}