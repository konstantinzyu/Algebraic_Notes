% Полупрямое произведение групп

\begin{theorem} \label{semidirect product}
	Следующие утверждения о группах $ G, N $ и $ H $ эквивалентны:
	\begin{enumerate}
		\item имеется расщепляющаяся точная последовательность
		$$ \begin{tikzcd}
		1 \ar{r}{} & N \ar{r} & G \ar{r} & H \ar{r} \ar[dashed, bend right=30]{l}[above]{s} & 1;
		\end{tikzcd} $$
		
		\item в группе $ G $ имеются подгруппы $ N' $ и $ H' $, изоморфные $ N $ и $ H $, соответственно, 
		такие, что $ N' \lhd G $, $ N'H' = G $ и $ N' \cap H' = \{e\} $;
		
		\item существует такой гомоморфизм $ \varphi \colon H \to \Aut{N} $,
		что группа $ G $ изоморфна группе пар $ (n, h) \in N \times H $ с умножением заданным по правилу
		$ (n_1, h_1)(n_2, h_2) = (n_1 \varphi(h_1)(n_2), h_1h_2) $. 
	\end{enumerate}
\end{theorem}

\begin{proof}
	\hfill
	
	\textbf{[$ 1 \Rightarrow 2 $]}
	
	Пусть $ i \colon N \to G $ --- вложение и $ \pi \colon G \to H $ --- проекция из точной последовательности.
	Положим $ N' = i(N) $ и $ H' = s(H) $. Так как $ N' = \ker \pi $, то $ N' $ нормальна в $ G $.
	Имеем $ s \circ \pi = \Id_H $ по определению расщепляющейся точной последовательности групп.
	Тогда $ H' \cap N' =  H' \cap \ker \pi = \{e\} $.
	Для всякого элемента $ g \in G $ имеем $ \pi(s(\pi(g))) = \pi(g) $.
	Положим $ h' = s(\pi(g)) $.
	Тогда $ gh'^{-1} \in \ker \pi = N' $ и поэтому $ G = N'H' $.
	
	\textbf{[$ 2 \Rightarrow 3 $]}
	
	Пусть $ f \colon N \to N',  g \colon H \to H' $ --- изоморфизмы.
	Зададим $ \varphi \colon H \to \Aut{N} $ по правилу $ \varphi \colon h \mapsto (n \mapsto f^{-1}(g(h)f(n)g(h)^{-1})) $.
	Обозначим через $ K $ группу пар $ (n, h) \in N \times H $ с указанным правилом умножения.
	Положим $ k \colon K \to G $ равным $ k \colon (n, h) \mapsto f(n)g(h) $.
	По построению $ \varphi $ отображение $ k $ является гомоморфизмом:
	$$ k\big((n_1,h_1)(n_2, h_2)\big) = k \big((n_1\varphi(h_1)(n_2), h_1h_2)\big)
	= k \big((n_1f^{-1}(g(h_1)f(n_2)g(h_1)^{-1}) , h_1h_2) \big) = $$
	$$ = f(n_1)g(h_1)f(n_2)g(h_1)^{-1}g(h_1h_2) = f(n_1)g(h_1)f(n_2)g(h_2)
	= k\big((n_1, h_1) \big) k\big((n_2, h_2) \big). $$
	
	\textbf{[$ 3 \Rightarrow 1 $]}
	
	Пусть, как в доказательстве предыдущей импликации $ K $ --- группа пар $ (n, h) \in N \times H $ с указанным правилом умножения и $ k \colon K \to G $ --- изоморфизм.
	
	Положим $ i \colon N \to G $ равным $ i \colon n \mapsto k\big((n, e_H) \big) $
	и $ \pi \colon G \to H $ равным $ \pi = \pr{H} \circ k^{-1} $,
	где $ \pr{H} $ --- гомоморфизм проекции $ K $ на $ H $.
	Как композиция инъекции и изоморфизма $ i $ инъективно,
	аналогично $ \pi $ сюръективно как композиции изоморфизма и сюръекции.
	По построению $ \Image i \subset \ker \pi $.
	Если $ g \in \ker \pi $, то $ k^{-1} = (n, e_H) $ для некоторого $ n \in N $.
	Тогда $ i(n) = g $, поэтому $ \ker \pi \subset \Image i $ и последовательность ниже точна:
	$$ \begin{tikzcd}
	1 \ar{r}{} & N \ar{r}[above]{i} & G \ar{r}[above]{\pi} & H \ar{r} & 1.
	\end{tikzcd} $$
	
	Положим $ s \colon H \to G $ равным $ s \colon h \mapsto k\big((e_N, h) \big) $.
	По построению $ \pi \circ s = \Id_H $ и последовательность расщепляется.
	
\end{proof}

Для двух групп $ N $ и $ H $ группа $ G $, удовлетворяющая одному из условий теоремы называется \textcolor{defcolor}{полупрямым произведением} групп $ N $ и $ H $ и обозначается $ N \leftthreetimes_{\varphi} H $ или $ N \rtimes_{\varphi} H $,
где $ \varphi \colon H \to \Aut{N} $ --- гомоморфизм из пункта 3. Если $ \varphi $ известно из контекста, то этот индекс опускается.

\begin{theorem} \label {pq groups}
	Пусть $ p < q $ --- простые числа. Тогда группа $ G $ порядка $ pq $ является полупрямым произведением
	$ \mathbb{Z}/q\mathbb{Z} \rtimes_{\varphi} \mathbb{Z}/p\mathbb{Z} $, где
	$ \varphi \colon \mathbb{Z}/p\mathbb{Z} \to \Aut{\mathbb{Z}/q\mathbb{Z}} \cong \mathbb{Z}/(q - 1)\mathbb{Z} $ --- гомоморфизм. Более того, 
	либо группа $ G $ изоморфна группе $ \mathbb{Z}/p\mathbb{Z} \oplus \mathbb{Z}/q\mathbb{Z} $,
	либо $ p $ делит $ q - 1 $ и группа $ G $ изоморфна группе, заданной образующими и соотношениями
	как $ \left< x, y \mid x^q, y^p, yxy^{-1}x^{-m} \right> $ для всякого $ m = r^{\tfrac{q - 1}{p}} $, где $ r $ --- первообразный корень по модулю $ q $.
\end{theorem}

\begin{proof}
	По теореме Коши в группе $ G $ порядка $ pq $ существует элемент $ a $ порядка $ q $ и элемент $ b $ порядка $ p $.
	Так как $ p $ --- наименьшее простое число, делящее порядок $ G $, 
	то подгруппа $ \langle a \rangle $ нормальна в $ G $.
	Факторгруппа $ G / \langle a \rangle $ изоморфна $ \mathbb{Z}/p\mathbb{Z} $.
	Отображение образующей группы $ \mathbb{Z}/p\mathbb{Z} $ 
	в элемент $ b $ индуцирует сечение $ s \colon \mathbb{Z}/p\mathbb{Z} \to G $
	точной последовательности
	$$ \begin{tikzcd}
	1 \ar{r}{} & \mathbb{Z}/q\mathbb{Z} \ar{r} & G \ar{r} & \mathbb{Z}/p\mathbb{Z} \ar{r} \ar[dashed, bend right=30]{l}[above]{s} & 1
	\end{tikzcd} $$
	поэтому $ G $ является полупрямым произведением $  \mathbb{Z}/q\mathbb{Z} \rtimes \mathbb{Z}/p\mathbb{Z} $.
	
	Группе $ G $ соответствует некоторый гомоморфизм $ \varphi \colon \mathbb{Z}/p\mathbb{Z} \to \Aut{\mathbb{Z}/q\mathbb{Z}} \cong \mathbb{Z}/(q - 1)\mathbb{Z} $ и наоборот, такому гомоморфизму соответствует группа некоторая $ G $. 
	В общем случае, это не биективное соответствие, но
	тривиальному гомоморфизму соответствует группа $ \mathbb{Z}/p\mathbb{Z} \oplus \mathbb{Z}/q\mathbb{Z} $
	и наоборот этой группе соответствует только тривиальный гомоморфизм.
	
	Если $ \varphi $ --- нетривиальный гомоморфизм, то $ p $ обязано делить $ \mathbb{Z}/q\mathbb{Z} $.
	Фиксируем первообразный корень $ r $ по модулю $ q $ и положим $ m = r^{\tfrac{q - 1}{p}} $
	и $ K = \left< x, y \mid x^q, y^p, yxy^{-1}x^{-m} \right> $.
	В силу соотношений, определяющих группу $ K $, каждый её элемент может быть представлен в виде $ x^sy^t $,
	где $ 0 \leqslant s < q $ и $ 0 \leqslant t < p $, поэтому группа $ K $ состоит не более чем из $ pq $ элементов.
	Построим сюръективный гомоморфизм $ f \colon K \to G $.
	Из сказанного выше, в группе $ G $ выполнено равенство $ bab^{-1} = a^m_1 $ для некоторого $ 1 < m_1 < q $,
	причём $ a = b^pab^{-p} = a^{(m_1^p)} $. Поэтому $ m_1^p \equiv 1 \mod q $. Так как $ m_1 \neq 1 $, 
	то оно является $ \tfrac{q - 1}{p} $-й степенью некоторого первообразного корня $ r_1 $ по модулю $ q $.
	Существует число $ n $ взаимно простое с $ q - 1 $ такое, что $ r_1^n = r $, поэтому $ m_1^n = m $ и 
	$ b^nab^{-n} = a^{m_1^n} = a^{m} $, элемент $ b^n $ имеет порядок $ p $.
	Определим гомоморфизм $ f $ на образующих по правилу $ f \colon x \mapsto a, y \mapsto b^n $.
	Так как $ G $ порождается элементами $ a $ и $ b^n $, то $ f $ сюръективно и, следовательно, изоморфизм.
	
\end{proof}