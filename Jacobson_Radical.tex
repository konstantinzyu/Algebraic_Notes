% Радикал ДжекобсОна

\begin{theorem}
	Пусть $ A $ --- ассоциативное кольцо с двусторонней единицей.
	Тогда следующие определения подмножества $ \Rad{A} $ эквивалентны:
	\begin{enumerate}
		\item $ \Rad{A} $ есть множество элементов $ a \in A $ таких, 
		что для всякого простого левого $ A $-модуля $ M $ \\ и $ m \in M $ выполнено $ am = 0 $;
		\item $ \Rad{A} $ есть пересечение всех максимальных левых идеалов;
		\item $ \Rad{A} = \defineset{a \in A}{\forall x,y \in A \ 1 - xay \text{ --- двусторонний обратимый}} $;
		\item $ \Rad{A} $ есть множество элементов $ a \in A $ таких, 
		что для всякого простого правого $ A $-модуля $ M $ и $ m \in M $ выполнено $ ma{\tiny } = 0 $;
		\item $ \Rad{A} $ есть пересечение всех максимальных правых идеалов.
	\end{enumerate}

	Множество $ \Rad{A} $ является двусторонним идеалом в $ A $.
\end{theorem}

\begin{proof}
	\hfill
	
	\textbf{[$ 1 \Leftrightarrow 2 $]}
	
	Согласно пункту 1 множество $ \Rad{A} $ равно пересечению аннуляторов всех простых левых $ A $-модулей.
	
	Пусть $ M $ --- простой левый $ A $-модуль и $ m \in M $.
	Докажем, что $ \Ann{m} $ --- максимальный левый идеал в $ A $.
	Действительно, пусть $ a \notin \Ann{m} $. Тогда имеем $ am = n \neq 0 $.
	Поскольку $ M $ прост, то $ An = M $ и найдётся $ b \in A $ такой, что
	$ bn = m $. Отсюда $ (1 - ba)m = m - bam = m - bn = m - m = 0 $ и $ 1 - ba \in \Ann{m} $.
	
	Так как $ \Ann{M} = \bigcap\limits_{m \in M} \Ann{m} $, то 
	имеем $$ \bigcap\limits_{M \text{--- простой}} \Ann{M} = 
	\bigcap\limits_{M \text{--- простой}} \bigcap\limits_{m \in M} \Ann{m} 
	\supset \bigcap_{I \text{--- макс. левый идеал}} I. $$
	С другой стороны, всякий максимальный левый идеал $ I $ является аннулятором элемента $ 1 + I $
	в левом простом модуле $ A / I $. 
	Поэтому последнее включение является равенством.
	
	\textbf{[$ 1 \Rightarrow \Rad{A} $ --- двусторонний идеал]}
	
	Аннулятор простого модуля является двусторонним идеалом, 
	пересечение двусторонних идеалов остаётся двусторонним идеалом.
	
	\textbf{[$ 4 \Leftrightarrow 5 $]}
	
	Следует из предыдущего для $ \op{A} $.
	
	\textbf{[$ 2 \Leftrightarrow 3 $]}
	
	Пусть $ a \in A $ таков, что $ 1 - xay $ является двусторонним обратимым для всяких $ x $ и $ y $.
	Если $ a $ не лежит в некотором максимальном левом идеале $ I $, то найдутся $ b \in A $ и $ i \in I $
	такие, что $ ba + i = 1 $. Однако, тогда $ i = 1 - ba $ обратим, что невозможно. 
	Противоречие показывает, что $ a $ лежит в пересечении всех максимальных левых идеалов.
	
	Пусть теперь $ r \in R = \bigcap\limits_{I \text{--- макс. левый идеал}} I $.
	Предположим, что у $ 1 - xry $ нет левого обратного.
	Тогда он содержится в некотором левом максимальном идеале $ I $.
	Однако, $ r \in R \subset I $ и $ R $ --- двусторонний идеал по доказанному выше.
	Поэтому $ ry \in R \subset I $ и $ xry \in I $. Отсюда следует, что $ 1 \in I $ и мы приходим к противоречию.
	
	Пусть теперь $ u(1 - xry) = 1 $.
	Тогда $ u = 1 - uxry $. По доказанному выше $ u $ имеет левый обратный $ v $.
	Следовательно, $ v = vu(1 - xry) = 1 - xry $. Поэтому $ (1 - xry)u = 1 $.
	
	\textbf{[$ 3 \Leftrightarrow 4 $]}
	
	Следует из предыдущего для $ \op{A} $.
	
	\textbf{[$ \Rad{A} $ --- двусторонний идеал]}
	
	Следует из определения, данного в пункте 1.
	
\end{proof}

Идеал $ \Rad{A} $ кольца $ A $ называется \textit{радикалом Джекобсона}.