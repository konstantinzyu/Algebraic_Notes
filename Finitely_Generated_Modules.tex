% \section{Классификация конечно порождённых модулей над областями главных идеалов}

\begin{theorem} \label{classification}
	Пусть $ R $ --- область главных идеалов и $ M $ --- конечно порождённый $ R $-модуль.
	Тогда существует однозначно определённое число $ s $ и 
	упорядоченный набор идеалов $ R \neq (d_1) \supset \ldots \supset (d_s) $ таких, что
	$$ M \cong \bigoplus\limits_{i = 1}^{s} R / (d_i). $$
	Кроме того, существует однозначно определённое число $ t $
	и набор (неупорядоченный) примарных идеалов $ \qq_i $
	такие, что
	$$ M \cong \bigoplus\limits_{i = 1}^{t} R / \qq_i. $$
\end{theorem}

\begin{lemma} \label{rank}
	Пусть $ R $ --- ненулевое коммутативное кольцо.
	Тогда изоморфизм $ R $-модулей $ R^k \cong R^m $ равносилен равенству $ k = m $.
\end{lemma}

\begin{proof}
	Из равенства показателей немедленно следует изоморфизм модулей. Докажем обратное.
	
	Так как $ R $ --- ненулевое кольцо, то оно обладает главным идеалом $ m $.
	Пусть $ \varphi \colon R^k \to R^m $ --- изоморфизм.
	Рассмотрим индуцированное отображение 
	$$ 1 \otimes \varphi \colon (R/m) \otimes R^k \to (R/m) \otimes R^m. $$
	Поскольку $ \varphi $ был изоморфизмом, то $ 1 \otimes \varphi $ также является изоморфизмом.
	Более того, $ 1 \otimes \varphi $ --- изоморфизм $ R/m $-модулей, 
	то есть векторных пространств над полем $ R/m $. Поэтому $ k = m $.
\end{proof}

Пусть $ M $ --- конечно порождённый свободный $ R $-модуль. 
\textcolor{defcolor}{Рангом} модуля $ M $ будем называть число образующих $ M $.
Согласно последней определение ранга корректно.

\begin{lemma} \label{free submodule}
	Пусть $ R $ --- область главных идеалов и $ M = R^k $ --- свободный $ R $-модуль.
	Пусть $ N \subset M $ --- его подмодуль. Тогда $ N $ свободно порождается над $ R $,
	причём ранг $ N $ не превосходит ранга $ M $.
\end{lemma}

\begin{proof}
	Докажем индукцией по $ k $.
	
	База индукции: $ k = 1 $.
	Пусть $ m \in M $ --- базисный элемент. Рассмотрим множество $ I = \defineset{r \in R}{rm \in N} $.
	Оно является идеалом в $ R $ и поэтому существует такое $ a \in R $, что $ I = (a) $.
	Тогда $ N $ порождается элементом $ n = am $. Если $ a = 0 $, то $ n = 0 $ и $ N $ свободен.
	Проверим, что $ \{n\} $ является базисом $ N $ в остальных случаях.
	Пусть для некоторого $ b \in R $ выполнено равенство $ bn = 0 $. Тогда $ 0 = bn = b(am) = (ba)m $
	и отсюда $ ba = 0 $. Так как $ R $ --- область целостности, то либо $ b = 0 $, либо $ a = 0 $.
	Последний случай был рассмотрен ранее, поэтому $ b = 0 $ и $ \{n\} $ --- базис $ N $.
	
	Шаг индукции.
	Пусть $ m_1, \ldots, m_k $ --- базис $ M $.
	Пусть $ M' = Rm_k $ --- свободный подмодуль, порождённый элементом $ m_k $.
	Пусть $ M'' = M/M' $. Тогда $ M'' $ является свободным модулем с $ k - 1 $ образующей:
	$ m_1 + Rm_k, \ldots, m_{k - 1} + Rm_k $.
	Пусть $ N $ --- подмодуль в $ M $. 
	Пусть $ N' = N \cap M' $ и $ N'' $ --- образ подмодуля $ N $ в $ M'' $
	при отображении факторизации.
	Имеем $ N/N' \cong N'' $.
	Оба модуля $ N' $ и $ N'' $ являются подмодулями конечно порождённых модулей.
	Поэтому, по индукционному предположению, $ N' $ и $ N'' $ являются свободными модулями.
	
	Если $ N' = 0 $, то $ N \cong N'' $ и теорема доказана.
	Пусть теперь $ n' $ --- базис модуля $ N' $
	и $ n_1'', \ldots, n_t'' $ --- базис модуля $ N'' $.
	Выберем в $ N $ по одному прообразу $ n_i $ для каждого базисного элемента модуля $ N'' $.
	Покажем, что $ n', n_1, \ldots, n_t $ порождают $ N $, а затем, что они являются базисом $ N $.
	
	Пусть $ n \in N $ и $ n'' $ --- его образ в $ N'' $.
	Тогда для некоторых $ a_1, \ldots, a_t \in R $ выполнено равенство
	$ n'' = a_1n_1'' + \ldots + a_tn_t'' $.
	Тогда разность $ n'' - (a_1n_1 + \ldots + a_tn_t) $ лежит в ядре проекции из $ N $ на $ N'' $,
	то есть в $ N' $. Поэтому $ n'' $ выражается через $ n', n_1, \ldots, n_t $.
	
	Проверим, что эти элементы образую базис $ N $.
	Пусть для некоторых $ a, a_1, \ldots, a_n \in R $ 
	выполнено $ an' + a_1n_1 + \ldots + a_tn_t = 0 $.
	Тогда в $ N'' $ выполнено $ a_1n_1'' + \ldots + a_tn_t'' = 0 $
	из чего следует, что $ a_1 = \ldots = a_t = 0 $, так как $ n_1'', \ldots, n_t'' $ --- базис $ N'' $.
	Тогда $ an' = 0 $ и уже $ a = 0 $, так как $ N' $ свободно порождён $ n' \neq 0 $.
	
	Из индукционного предположения следует, что ранг $ N $ не превосходит $ 1 + (k - 1) = k $.
\end{proof}

\begin{lemma}[Нормальная форма Смита] \label{submodule}
	Пусть $ R $ --- область главных идеалов и $ M = R^k $ --- свободный $ R $-модуль.
	Пусть $ N \subset M $ --- его подмодуль. 
	Тогда найдутся базис $ m_1, \ldots, m_k $ модуля M
	и элементы $ d_1, \ldots, d_k $ кольца $ R $ такие, что $ (d_1) \supset \ldots \supset (d_k) $
	и $ N $ свободно порождается
	ненулевыми элементами из набора $ d_1m_1, \ldots, d_km_k $.
	Более того $ d_1, \ldots, d_k $ определены однозначно с точностью до умножения на обратимые элементы кольца.
\end{lemma}

\begin{proof}
	Пусть $ x_1, \ldots, x_k $ --- базис $ M $
	и $ y_1, \ldots, y_s $ --- базис $ N $.
	Элементы $ y_1, \ldots, y_s $ выражаются как линейные комбинации базисных $ x_1, \ldots, x_k $.
	Запишем в матрицу $ C $ размера $ k \times s $ коэффициенты, с которыми базисные $ x_i $ 
	входят в разложение элементов $ y_j $.
	Обратимые элементарные преобразования строк и столбцом матрицы $ C $ 
	соответствуют замене базиса в $ M $ или $ N $.
	
	Пусть $ a_{11}, \ldots, a_{1s} $ --- первая строка матрицы.
	Можем считать, что она ненулевая, иначе поменяем её местами с ненулевой строкой.
	Так же можем считать, что $ a_{11} \neq 0 $, иначе поменяем столбцы.
	Пусть $ (a) = (a_{11}, a_{12}) $ --- идеал в $ R $.
	Тогда существуют $ r_1, r_2, q_1, q_2 \in R $ такие, что
	$ a = r_1a_{11} + r_2a_{12} $ и $ a_{11} = q_1a, a_{12} = q_2a $.
	Отсюда $ r_1q_1 + r_2q_2 = 1 $.
	Тогда следующая матрица является обратимой:
	$$ \begin{pmatrix}
	r_1 & -q_2 & 0 & \ldots & 0 \\
	r_2 & q_1  & 0 & \ldots & 0 \\
	0   & 0    & 1 & \ldots & 0 \\
	\vdots & \vdots & \vdots & \ddots & \vdots \\
	\end{pmatrix} $$
	Умножим матрицу перехода $ C $ на эту матрицу. 
	В новой матрице на позиции $ (1, 1) $ будет стоять элемент $ a $.
	Повторив операцию для элементов на позициях $ (1, 1) $ и $ (1, 3) $ и далее
	мы добьёмся того, чтобы на позиции $ (1, 1) $ стоял НОД всей первой строки исходной матрицы.
	Элементарными преобразованиями строк сделаем нулевыми все остальные элементы строки.
	Проделаем аналогичную операцию с первым столбцом.
	Будем повторять процесс до тех пор, 
	пока и первая строка и первый столбец не будут содержать единственный ненулевой элемент на позиции $ (1, 1) $.
	Процесс завершится за конечное время, 
	так как возрастающая цепочка идеалов, порождённых элементом на позиции $ (1, 1) $ стабилизируется.
	
	Повторим операцию, описанную выше, для подматрицы $ (k - 1) \times (s - 1) $,
	полученной удалением первых строки и столбца.
	Далее, будем повторять операцию до тех пор, пока не образуется диагональная матрица
	$$ \begin{pmatrix}
	a_1 & 0 & 0 & \ldots & 0 \\
	0 & a_2  & 0 & \ldots & 0 \\
	0   & 0    & a_3 & \ldots & 0 \\
	\vdots & \vdots & \vdots & \ddots & \vdots \\
	\end{pmatrix} $$
	Теперь покажем, как из этой матрицы получить матрицу, 
	в которой каждый элемент на диагонали делит следующий.
	Пусть снова $ (a) = (a_{1}, a_2) $ и $ a = r_1a_1 + r_2a_2, a_1 = q_1a, a_2 = q_2a $.
	Тогда $ r_1q_1 + r_2q_2 = 1 $. 
	Будем выполнять элементарные преобразования только первых двух строк и столбцов:
	$$ \begin{pmatrix}
	a_1 & 0  \\
	0 & a_2   \\
	\end{pmatrix}
	\rightsquigarrow
	\begin{pmatrix}
		a_1 & a_2  \\
		0 & a_2   \\
	\end{pmatrix}
	\rightsquigarrow
	\begin{pmatrix}
	a_1 & a_2  \\
	0 & a_2   \\
	\end{pmatrix}
	\begin{pmatrix}
	r_1 & -q_2  \\
	r_2 & q_1   \\
	\end{pmatrix}
	=
	\begin{pmatrix}
	a & -q_2a_1 + q_1a_2  \\
	r_2a_2 & q_1a_2   \\
	\end{pmatrix}
	\rightsquigarrow
	\begin{pmatrix}
	a & 0  \\
	0 & *   \\
	\end{pmatrix} $$
	При всех преобразованиях определитель сохранился, а на позиции $ (2, 2) $ теперь стоит элемент, кратный $ a $.
	Повторяя эту операцию для всех пар диагональных элементов мы добьёмся требуемого.
	Коэффициентами $ d_i $ будут служить элементы на диагонали, а в качестве базиса $ m_i $
	следует взять получившийся в ходе замены базис $ M $.
	
	При умножении на матрицы в ходе рассуждения выше 
	наибольший общий делитель миноров каждого фиксированного размера 
	продолжал делиться на НОД миноров того же размера до умножения.
	Так как матрицы и преобразования были обратимы, 
	то наибольший общий делитель всех миноров фиксированного размера 
	не изменится в ходе преобразований.
	Поэтому элементы $ d_i $ определены однозначно с точностью до умножения на обратимые элементы кольца $ R $.
\end{proof}

\begin{proof}[Доказательство теоремы  о классификации]
	Пусть $ M $ --- $ k $-порождённый $ R $-модуль.
	Пусть $ m_1, \ldots, m_k $ --- образующие $ M $.
	Накроем $ M $ свободным модулем $ R^k $ с образующими $ x_1, \ldots, x_k $,
	посредством отображения, сопоставляющее элементу $ x_i $ элемент $ m_i $.
	Пусть $ N $ --- ядро этого гомоморфизма.
	
	По лемме о нормальной форме Смита существует базис $ y_1, \ldots, y_k $
	и определённые с точностью до умножения на константу элементы $ d_1 \mid \ldots \mid d_k $
	такие, что $ d_1y_1, \ldots d_ky_k $ свободно порождают подмодуль $ N $.
	Удалим все обратимые $ d_i $ и перенумеруем их.
	Тогда $ M \cong R^k/N \cong \bigoplus\limits_{i = 0}^{s} R/(d_i) $.
	Идеалы $ R \neq (d_1) \supset \ldots (d_s) $ определены однозначно.
	
	По китайской теореме об остатках и факториальности кольца $ R $ 
	модуль $ R/(d_i) $ однозначно раскладывается в прямую сумму модулей вида $ R/(p_j^{a_j}) $,
	где $ p_j $ --- простой элемент.
	Единственность разложения $ M $ 
	следует из возможности восстановить $ d_i $ по разложению на факторы по примарным идеалам.
	
\end{proof}